\documentclass{article}
\usepackage[elliptic]{qed}

\begin{document}

\section*{Numerical fallback}

The Legendre relation is a relation among the elliptic integrals of the first ($\ellipk$) and second
($\ellipe$) kinds, given by
%
\begin{align*}
    \ellipe(k^2) \ellipk(1-k^2) + \ellipe(1 - k^2) \ellipk(k^2) - \ellipk(k^2) \ellipk(1 - k^2) = \frac{\pi}{2}
\end{align*}
%
Unfortunately, \textsf{sympy} doesn't know how to automatically simplify it, so \textsf{qed} cannot
analytically determine its validity:
%
\begin{qed}[numerical=no]
    \ellipe(k^2) \ellipk(1-k^2) + \ellipe(1 - k^2) \ellipk(k^2) - \ellipk(k^2) \ellipk(1 - k^2) = \frac{\pi}{2}
\end{qed}
%
Notice that in the above expression we passed \texttt{numerical=no} to the \texttt{qed} environment.
If instead we pass \texttt{numerical=fallback}, which is the default, we find that \textsf{qed} correctly
validates the equation:
%
\begin{qed}[numerical=fallback]
    \ellipe(k^2) \ellipk(1-k^2) + \ellipe(1 - k^2) \ellipk(k^2) - \ellipk(k^2) \ellipk(1 - k^2) = \frac{\pi}{2}
\end{qed}
%
By default, \textsf{qed} randomizes values of all the free variables in the range
$[\qedDefaultQedLow, \qedDefaultQedHigh]$ and evaluates the expression \texttt{\qedDefaultQedNTests} times.
This is not always what we want, so we can provide custom values for the \texttt{low},
\texttt{high}, \texttt{ntests}, and randomizer \texttt{seed} parameters:
%
\begin{qed}[seed=0, low=-10.0, high=10.0, ntests=100]
    \ellipe(k^2) \ellipk(1-k^2) + \ellipe(1 - k^2) \ellipk(k^2) - \ellipk(k^2) \ellipk(1 - k^2) = \frac{\pi}{2}
\end{qed}
%
Alternatively, we can specify the values of the variables directly using the \texttt{variables}
keyword, whose value should be a JSON-like dictionary with keys for each free variable in the
expression:
%
\begin{qed}[variables={"k": 0.5}]
    \ellipe(k^2) \ellipk(1-k^2) + \ellipe(1 - k^2) \ellipk(k^2) - \ellipk(k^2) \ellipk(1 - k^2) = \frac{\pi}{2}
\end{qed}
%
Arrays of values may also be provided:
%
\begin{qed}[variables={"k": [0.1, 0.2, 0.3, 0.4, 0.5]}]
    \ellipe(k^2) \ellipk(1-k^2) + \ellipe(1 - k^2) \ellipk(k^2) - \ellipk(k^2) \ellipk(1 - k^2) = \frac{\pi}{2}
\end{qed}
%
Note that if there are multiple free variables, all arrays must have the same length.
Finally, simple \textsf{Python} expressions are also allowed (these should be provided
as strings):
%
\begin{qed}[variables={"k": "np.linspace(0.01, 0.99, 100)"}]
    \ellipe(k^2) \ellipk(1-k^2) + \ellipe(1 - k^2) \ellipk(k^2) - \ellipk(k^2) \ellipk(1 - k^2) = \frac{\pi}{2}
\end{qed}
%
Note that only the \textsf{numpy} and \textsf{sympy} packages are guaranteed to
be imported into the namespace.

\end{document}
