\documentclass{article}
\usepackage[elliptic]{qed}

\begin{document}

\section*{Elliptic integrals}

The complete elliptic integrals of the first and second kinds are defined as
%
\begin{qed}
    \ellipk(k^2) = \int_0^{\frac{\pi}{2}}
    \frac{1}{\sqrt{1 - k^2 \sin^2\theta}} \, \dd \theta
\end{qed}
%
and
%
\begin{qed}
    \ellipe(k^2) = \int_0^{\frac{\pi}{2}}
    \sqrt{1 - k^2 \sin^2\theta} \, \dd \theta
\end{qed}
%
respectively. They have the following properties:
%
\begin{qed}
    \ellipk(0) = \frac{\pi}{2}
\end{qed}
%
\begin{qed}
    \ellipe(0) = \frac{\pi}{2}
\end{qed}
%
\begin{qed}
    \ellipk(1) = \cinfty
\end{qed}
%
\begin{qed}
    \ellipe(1) = 1
\end{qed}
%
\begin{qed}
    \frac{d}{d k}\ellipk(k^2) = -\frac{\ellipk(k^2)
        + \frac{\ellipe(k^2)}{k^2 - 1}}{k}
\end{qed}
%
\begin{qed}
    \frac{d}{d k}\ellipe(k^2) = \frac{\ellipe(k^2) - \ellipk(k^2)}{k}
\end{qed}
%
They also satisfy the Legendre relation:
%
\begin{qed}
    \ellipe(k^2) \ellipk(1-k^2) + \ellipe(1 - k^2) \ellipk(k^2)
    - \ellipk(k^2) \ellipk(1 - k^2) = \frac{\pi}{2}
\end{qed}
%
\end{document}
